%% bare_conf_compsoc.tex
%% V1.4b
%% 2015/08/26
%% by Michael Shell
%% See:
%% http://www.michaelshell.org/
%% for current contact information.
%%
%% This is a skeleton file demonstrating the use of IEEEtran.cls
%% (requires IEEEtran.cls version 1.8b or later) with an IEEE Computer
%% Society conference paper.
%%
%% Support sites:
%% http://www.michaelshell.org/tex/ieeetran/
%% http://www.ctan.org/pkg/ieeetran
%% and
%% http://www.ieee.org/

%%*************************************************************************
%% Legal Notice:
%% This code is offered as-is without any warranty either expressed or
%% implied; without even the implied warranty of MERCHANTABILITY or
%% FITNESS FOR A PARTICULAR PURPOSE! 
%% User assumes all risk.
%% In no event shall the IEEE or any contributor to this code be liable for
%% any damages or losses, including, but not limited to, incidental,
%% consequential, or any other damages, resulting from the use or misuse
%% of any information contained here.
%%
%% All comments are the opinions of their respective authors and are not
%% necessarily endorsed by the IEEE.
%%
%% This work is distributed under the LaTeX Project Public License (LPPL)
%% ( http://www.latex-project.org/ ) version 1.3, and may be freely used,
%% distributed and modified. A copy of the LPPL, version 1.3, is included
%% in the base LaTeX documentation of all distributions of LaTeX released
%% 2003/12/01 or later.
%% Retain all contribution notices and credits.
%% ** Modified files should be clearly indicated as such, including  **
%% ** renaming them and changing author support contact information. **
%%*************************************************************************


% *** Authors should verify (and, if needed, correct) their LaTeX system  ***
% *** with the testflow diagnostic prior to trusting their LaTeX platform ***
% *** with production work. The IEEE's font choices and paper sizes can   ***
% *** trigger bugs that do not appear when using other class files.       ***                          ***
% The testflow support page is at:
% http://www.michaelshell.org/tex/testflow/



\documentclass[conference,compsoc]{IEEEtran}
% Some/most Computer Society conferences require the compsoc mode option,
% but others may want the standard conference format.
%
% If IEEEtran.cls has not been installed into the LaTeX system files,
% manually specify the path to it like:
% \documentclass[conference,compsoc]{../sty/IEEEtran}





% Some very useful LaTeX packages include:
% (uncomment the ones you want to load)


% *** MISC UTILITY PACKAGES ***
%
%\usepackage{ifpdf}
% Heiko Oberdiek's ifpdf.sty is very useful if you need conditional
% compilation based on whether the output is pdf or dvi.
% usage:
% \ifpdf
%   % pdf code
% \else
%   % dvi code
% \fi
% The latest version of ifpdf.sty can be obtained from:
% http://www.ctan.org/pkg/ifpdf
% Also, note that IEEEtran.cls V1.7 and later provides a builtin
% \ifCLASSINFOpdf conditional that works the same way.
% When switching from latex to pdflatex and vice-versa, the compiler may
% have to be run twice to clear warning/error messages.






% *** CITATION PACKAGES ***
%
\ifCLASSOPTIONcompsoc
  % IEEE Computer Society needs nocompress option
  % requires cite.sty v4.0 or later (November 2003)
  \usepackage[nocompress]{cite}
\else
  % normal IEEE
  \usepackage{cite}
\fi

\usepackage{tabularx}
\usepackage{booktabs}
\usepackage{amsmath}
\usepackage{url}
\usepackage{placeins}
% cite.sty was written by Donald Arseneau
% V1.6 and later of IEEEtran pre-defines the format of the cite.sty package
% \cite{} output to follow that of the IEEE. Loading the cite package will
% result in citation numbers being automatically sorted and properly
% "compressed/ranged". e.g., [1], [9], [2], [7], [5], [6] without using
% cite.sty will become [1], [2], [5]--[7], [9] using cite.sty. cite.sty's
% \cite will automatically add leading space, if needed. Use cite.sty's
% noadjust option (cite.sty V3.8 and later) if you want to turn this off
% such as if a citation ever needs to be enclosed in parenthesis.
% cite.sty is already installed on most LaTeX systems. Be sure and use
% version 5.0 (2009-03-20) and later if using hyperref.sty.
% The latest version can be obtained at:
% http://www.ctan.org/pkg/cite
% The documentation is contained in the cite.sty file itself.
%
% Note that some packages require special options to format as the Computer
% Society requires. In particular, Computer Society  papers do not use
% compressed citation ranges as is done in typical IEEE papers
% (e.g., [1]-[4]). Instead, they list every citation separately in order
% (e.g., [1], [2], [3], [4]). To get the latter we need to load the cite
% package with the nocompress option which is supported by cite.sty v4.0
% and later.





% *** GRAPHICS RELATED PACKAGES ***
%
\ifCLASSINFOpdf
  % \usepackage[pdftex]{graphicx}
  % declare the path(s) where your graphic files are
  % \graphicspath{{../pdf/}{../jpeg/}}
  % and their extensions so you won't have to specify these with
  % every instance of \includegraphics
  % \DeclareGraphicsExtensions{.pdf,.jpeg,.png}
\else
  % or other class option (dvipsone, dvipdf, if not using dvips). graphicx
  % will default to the driver specified in the system graphics.cfg if no
  % driver is specified.
  % \usepackage[dvips]{graphicx}
  % declare the path(s) where your graphic files are
  % \graphicspath{{../eps/}}
  % and their extensions so you won't have to specify these with
  % every instance of \includegraphics
  % \DeclareGraphicsExtensions{.eps}
\fi
% graphicx was written by David Carlisle and Sebastian Rahtz. It is
% required if you want graphics, photos, etc. graphicx.sty is already
% installed on most LaTeX systems. The latest version and documentation
% can be obtained at: 
% http://www.ctan.org/pkg/graphicx
% Another good source of documentation is "Using Imported Graphics in
% LaTeX2e" by Keith Reckdahl which can be found at:
% http://www.ctan.org/pkg/epslatex
%
% latex, and pdflatex in dvi mode, support graphics in encapsulated
% postscript (.eps) format. pdflatex in pdf mode supports graphics
% in .pdf, .jpeg, .png and .mps (metapost) formats. Users should ensure
% that all non-photo figures use a vector format (.eps, .pdf, .mps) and
% not a bitmapped formats (.jpeg, .png). The IEEE frowns on bitmapped formats
% which can result in "jaggedy"/blurry rendering of lines and letters as
% well as large increases in file sizes.
%
% You can find documentation about the pdfTeX application at:
% http://www.tug.org/applications/pdftex





% *** MATH PACKAGES ***
%
%\usepackage{amsmath}
% A popular package from the American Mathematical Society that provides
% many useful and powerful commands for dealing with mathematics.
%
% Note that the amsmath package sets \interdisplaylinepenalty to 10000
% thus preventing page breaks from occurring within multiline equations. Use:
%\interdisplaylinepenalty=2500
% after loading amsmath to restore such page breaks as IEEEtran.cls normally
% does. amsmath.sty is already installed on most LaTeX systems. The latest
% version and documentation can be obtained at:
% http://www.ctan.org/pkg/amsmath





% *** SPECIALIZED LIST PACKAGES ***
%
%\usepackage{algorithmic}
% algorithmic.sty was written by Peter Williams and Rogerio Brito.
% This package provides an algorithmic environment fo describing algorithms.
% You can use the algorithmic environment in-text or within a figure
% environment to provide for a floating algorithm. Do NOT use the algorithm
% floating environment provided by algorithm.sty (by the same authors) or
% algorithm2e.sty (by Christophe Fiorio) as the IEEE does not use dedicated
% algorithm float types and packages that provide these will not provide
% correct IEEE style captions. The latest version and documentation of
% algorithmic.sty can be obtained at:
% http://www.ctan.org/pkg/algorithms
% Also of interest may be the (relatively newer and more customizable)
% algorithmicx.sty package by Szasz Janos:
% http://www.ctan.org/pkg/algorithmicx




% *** ALIGNMENT PACKAGES ***
%
%\usepackage{array}
% Frank Mittelbach's and David Carlisle's array.sty patches and improves
% the standard LaTeX2e array and tabular environments to provide better
% appearance and additional user controls. As the default LaTeX2e table
% generation code is lacking to the point of almost being broken with
% respect to the quality of the end results, all users are strongly
% advised to use an enhanced (at the very least that provided by array.sty)
% set of table tools. array.sty is already installed on most systems. The
% latest version and documentation can be obtained at:
% http://www.ctan.org/pkg/array


% IEEEtran contains the IEEEeqnarray family of commands that can be used to
% generate multiline equations as well as matrices, tables, etc., of high
% quality.




% *** SUBFIGURE PACKAGES ***
%\ifCLASSOPTIONcompsoc
%  \usepackage[caption=false,font=footnotesize,labelfont=sf,textfont=sf]{subfig}
%\else
%  \usepackage[caption=false,font=footnotesize]{subfig}
%\fi
% subfig.sty, written by Steven Douglas Cochran, is the modern replacement
% for subfigure.sty, the latter of which is no longer maintained and is
% incompatible with some LaTeX packages including fixltx2e. However,
% subfig.sty requires and automatically loads Axel Sommerfeldt's caption.sty
% which will override IEEEtran.cls' handling of captions and this will result
% in non-IEEE style figure/table captions. To prevent this problem, be sure
% and invoke subfig.sty's "caption=false" package option (available since
% subfig.sty version 1.3, 2005/06/28) as this is will preserve IEEEtran.cls
% handling of captions.
% Note that the Computer Society format requires a sans serif font rather
% than the serif font used in traditional IEEE formatting and thus the need
% to invoke different subfig.sty package options depending on whether
% compsoc mode has been enabled.
%
% The latest version and documentation of subfig.sty can be obtained at:
% http://www.ctan.org/pkg/subfig




% *** FLOAT PACKAGES ***
%
%\usepackage{fixltx2e}
% fixltx2e, the successor to the earlier fix2col.sty, was written by
% Frank Mittelbach and David Carlisle. This package corrects a few problems
% in the LaTeX2e kernel, the most notable of which is that in current
% LaTeX2e releases, the ordering of single and double column floats is not
% guaranteed to be preserved. Thus, an unpatched LaTeX2e can allow a
% single column figure to be placed prior to an earlier double column
% figure.
% Be aware that LaTeX2e kernels dated 2015 and later have fixltx2e.sty's
% corrections already built into the system in which case a warning will
% be issued if an attempt is made to load fixltx2e.sty as it is no longer
% needed.
% The latest version and documentation can be found at:
% http://www.ctan.org/pkg/fixltx2e


%\usepackage{stfloats}
% stfloats.sty was written by Sigitas Tolusis. This package gives LaTeX2e
% the ability to do double column floats at the bottom of the page as well
% as the top. (e.g., "\begin{figure*}[!b]" is not normally possible in
% LaTeX2e). It also provides a command:
%\fnbelowfloat
% to enable the placement of footnotes below bottom floats (the standard
% LaTeX2e kernel puts them above bottom floats). This is an invasive package
% which rewrites many portions of the LaTeX2e float routines. It may not work
% with other packages that modify the LaTeX2e float routines. The latest
% version and documentation can be obtained at:
% http://www.ctan.org/pkg/stfloats
% Do not use the stfloats baselinefloat ability as the IEEE does not allow
% \baselineskip to stretch. Authors submitting work to the IEEE should note
% that the IEEE rarely uses double column equations and that authors should try
% to avoid such use. Do not be tempted to use the cuted.sty or midfloat.sty
% packages (also by Sigitas Tolusis) as the IEEE does not format its papers in
% such ways.
% Do not attempt to use stfloats with fixltx2e as they are incompatible.
% Instead, use Morten Hogholm'a dblfloatfix which combines the features
% of both fixltx2e and stfloats:
%
% \usepackage{dblfloatfix}
% The latest version can be found at:
% http://www.ctan.org/pkg/dblfloatfix




% *** PDF, URL AND HYPERLINK PACKAGES ***
%
%\usepackage{url}
% url.sty was written by Donald Arseneau. It provides better support for
% handling and breaking URLs. url.sty is already installed on most LaTeX
% systems. The latest version and documentation can be obtained at:
% http://www.ctan.org/pkg/url
% Basically, \url{my_url_here}.




% *** Do not adjust lengths that control margins, column widths, etc. ***
% *** Do not use packages that alter fonts (such as pslatex).         ***
% There should be no need to do such things with IEEEtran.cls V1.6 and later.
% (Unless specifically asked to do so by the journal or conference you plan
% to submit to, of course. )


% correct bad hyphenation here
\hyphenation{op-tical net-works semi-conduc-tor}
\usepackage{graphicx}
\graphicspath{{results/}{results/rlchef/}}



\begin{document}
%
% paper title
% Titles are generally capitalized except for words such as a, an, and, as,
% at, but, by, for, in, nor, of, on, or, the, to and up, which are usually
% not capitalized unless they are the first or last word of the title.
% Linebreaks \\ can be used within to get better formatting as desired.
% Do not put math or special symbols in the title.
\title{RL Chef: Tabular Reinforcement Learning vs Linear Function Approximation \\ and the Impact of State Representation}


% author names and affiliations
% use a multiple column layout for up to three different
% affiliations
\author{\IEEEauthorblockN{Luigi Daddario (mat. 908294)}
\IEEEauthorblockA{Artificial Intelligence For Science And Technology\\
University of Milano-Bicocca\\
Email: l.daddario1@campus.unimib.it}}



% conference papers do not typically use \thanks and this command
% is locked out in conference mode. If really needed, such as for
% the acknowledgment of grants, issue a \IEEEoverridecommandlockouts
% after \documentclass

% for over three affiliations, or if they all won't fit within the width
% of the page (and note that there is less available width in this regard for
% compsoc conferences compared to traditional conferences), use this
% alternative format:
% 
%\author{\IEEEauthorblockN{Michael Shell\IEEEauthorrefmark{1},
%Homer Simpson\IEEEauthorrefmark{2},
%James Kirk\IEEEauthorrefmark{3}, 
%Montgomery Scott\IEEEauthorrefmark{3} and
%Eldon Tyrell\IEEEauthorrefmark{4}}
%\IEEEauthorblockA{\IEEEauthorrefmark{1}School of Electrical and Computer Engineering\\
%Georgia Institute of Technology,
%Atlanta, Georgia 30332--0250\\ Email: see http://www.michaelshell.org/contact.html}
%\IEEEauthorblockA{\IEEEauthorrefmark{2}Twentieth Century Fox, Springfield, USA\\
%Email: homer@thesimpsons.com}
%\IEEEauthorblockA{\IEEEauthorrefmark{3}Starfleet Academy, San Francisco, California 96678-2391\\
%Telephone: (800) 555--1212, Fax: (888) 555--1212}
%\IEEEauthorblockA{\IEEEauthorrefmark{4}Tyrell Inc., 123 Replicant Street, Los Angeles, California 90210--4321}}




% use for special paper notices
%\IEEEspecialpapernotice{(Invited Paper)}




% make the title area
\maketitle

% As a general rule, do not put math, special symbols or citations
% in the abstract
\begin{abstract}
We study a custom Gymnasium environment, \emph{RL Chef}, where an agent navigates a $5\times5$ grid, collects ingredients under supply constraints, and decides when to cook a dish to maximize reward. We compare tabular control (Q-learning) with linear function approximation for the action-value function, and we analyze how state representation affects learning and convergence. In particular, we contrast an intentionally lossy, non-Markov representation (\texttt{simple}) with a Markov representation that augments the state with a visited-cells mask (\texttt{mask}). Across multiple random seeds we report learning curves, final evaluation return, and interpretable domain metrics (waste and incompatible ingredient usage).
\end{abstract}

% no keywords




% For peer review papers, you can put extra information on the cover
% page as needed:
% \ifCLASSOPTIONpeerreview
% \begin{center} \bfseries EDICS Category: 3-BBND \end{center}
% \fi
%
% For peerreview papers, this IEEEtran command inserts a page break and
% creates the second title. It will be ignored for other modes.
\IEEEpeerreviewmaketitle



\section{Introduction}
Reinforcement learning (RL) studies how an agent can learn to act optimally through interaction with an environment modeled as a Markov Decision Process (MDP) \cite{SuttonBarto2018}. In small discrete domains, tabular methods such as Q-learning can converge to the optimal action-value function under standard assumptions \cite{Watkins1992}. In larger state spaces, value function approximation is typically required to enable generalization \cite{SuttonBarto2018}.

This report focuses on \emph{RL Chef}, a compact gridworld designed to highlight two key themes: (i) the trade-off between tabular control and linear function approximation, and (ii) the critical role of state representation and Markovity. The environment includes delayed decision-making (choosing when to cook), penalties for waste, and constraints induced by one-time ingredient availability, creating a non-trivial exploration--exploitation problem even on a small grid.


\section{Objectives}
The objectives of this project are:
\begin{itemize}
\item Model the cooking task as an MDP and implement it with the Gymnasium API \cite{Gymnasium}.
\item Implement and compare:
  \begin{itemize}
  \item tabular Q-learning (and optionally SARSA);
  \item linear action-value approximation trained with temporal-difference updates.
  \end{itemize}
\item Analyze the impact of state representation by comparing an intentionally lossy, non-Markov representation (\texttt{simple}) with a Markov representation that augments the state with a visited-cells mask (\texttt{mask}).
\item Produce reproducible results across multiple seeds, including learning curves, evaluation metrics, and domain-specific diagnostics.
\end{itemize}


\section{Methods}

\subsection{Dataset}
No external dataset is used. All experience is generated online by interacting with the environment and collecting trajectories of $(s_t, a_t, r_t, s_{t+1})$.


\subsection{Optimization and Regularization}
We use $\epsilon$-greedy exploration with an optional linear decay schedule $\epsilon_t$ from $\epsilon_0$ to $\epsilon_{\mathrm{final}}$ over a fixed number of episodes. The discount factor is fixed to $\gamma \in (0,1)$. For tabular control we update a Q-table (Q-learning update); for linear approximation we parameterize:
\[
Q_\theta(s,a) = w_a^\top \phi(s) + b_a
\]
and perform semi-gradient temporal-difference updates on the parameters. L2 regularization is available in the implementation but is set to $0$ in the main experiments.

\subsection{Environment and State Representations}
The environment is a $5\times5$ grid. Each cell contains one ingredient type; ingredients can be collected at most once per episode (supply constraint). The agent chooses among 5 actions: 4-direction movement plus \texttt{cook}. The \texttt{cook} action ends the episode and yields a reward based on the best feasible recipe given the inventory, minus penalties for waste and incompatible ingredient combinations.

\begin{figure}[t]
\centering
\includegraphics[width=\linewidth]{rlchef/env_layout.png}
\caption{Static overview of the fixed grid layout used in experiments (ingredient id and short name per cell).}
\label{fig:env_layout}
\end{figure}

\begin{table}[t]
\centering
\caption{Environment parameters and recipe definitions (auto-generated from code).}
\label{tab:env_specs}
% Auto-generated by: python -m rlchef.analyze --extra-figs
\begin{tabular}{ll}
\toprule
Parameter & Value \\
\midrule
Grid size & 5$\times$5 \\
Actions & 5 (up, down, left, right, cook) \\
Max steps & 60 \\
Pickup reward & 0.05 \\
Move penalty & 0.01 \\
Waste penalty & 0.20 per wasted item \\
Incompatibility penalty & 0.30 per incompatible pair \\
Fail penalty (no feasible recipe) & 0.50 \\
Extra penalty (cook with empty inv.) & 0.70 \\
\midrule
\multicolumn{2}{l}{\textbf{Recipes (value; required ingredients)}}\\
margherita & 4.0 ; (1, 1, 0, 1, 0) \\
pasta_al_pomodoro & 3.5 ; (1, 0, 1, 1, 0) \\
cheesy_pasta & 3.0 ; (0, 1, 1, 0, 0) \\
fish_special & 3.2 ; (0, 0, 0, 1, 1) \\
\bottomrule
\end{tabular}

\end{table}

We compare two discrete state keys for tabular control:
\begin{itemize}
\item \textbf{\texttt{simple}}: position and inventory counts (lossy / non-Markov because it omits which cells have been depleted).
\item \textbf{\texttt{mask}}: augments \texttt{simple} with a bitmask of visited/depleted cells, restoring Markovity under a fixed grid layout.
\end{itemize}

Both agents also receive a continuous observation vector with normalized features (position, current-cell ingredient one-hot, availability, inventory, remaining supply; and optional budget/round features in the challenging variant).


\section{Experiments}
We run a benchmark that compares four configurations across multiple random seeds:
\begin{itemize}
\item Q-learning with \texttt{simple} state;
\item Q-learning with \texttt{mask} state;
\item linear function approximation with \texttt{simple} observation;
\item linear function approximation with \texttt{mask} observation.
\end{itemize}

Each run trains for 20{,}000 episodes with $\epsilon$-greedy exploration ($\epsilon_0=0.3 \rightarrow \epsilon_{\mathrm{final}}=0.05$ linearly over 3{,}000 episodes) and uses 10 random seeds. Final evaluation sets $\epsilon=0$ and estimates metrics over 1{,}000 evaluation episodes per run. We report the mean and standard deviation (across seeds) of evaluation return, average steps, and domain-specific diagnostics (waste and incompatibility). Additionally, we compute sample-efficiency proxies from the training curve: the average training return over all episodes (AUC proxy) and the mean return over the last episodes (stability / late performance).

\subsection{Implementation details and reproducibility}
All experiments can be reproduced by running \texttt{python -m rlchef.experiments} followed by \texttt{python -m rlchef.analyze}. Raw outputs are saved to \texttt{results/rlchef/}: \texttt{summaries.json} (final metrics per run), \texttt{curves.json} (per-episode learning curves), and the generated plots (\texttt{.png}) and tables (\texttt{.tex}) referenced throughout this report.

Compute cost is modest for this environment: on a standard macOS laptop, a single seed run with 20{,}000 training episodes (covering all four configurations) takes on the order of tens of seconds, and the 10-seed benchmark completes within minutes; plotting/aggregation adds negligible overhead relative to training.\footnote{Wall-clock time depends on hardware and Python environment; the goal here is to provide an order-of-magnitude estimate.}

\subsection{Results}
Table~\ref{tab:metrics} reports aggregate evaluation metrics across seeds. Q-learning achieves high return with low variability: $3.307\pm0.104$ for \texttt{simple} and $3.177\pm0.079$ for \texttt{mask}. In contrast, the linear baseline is considerably less consistent, achieving $1.652\pm0.718$ in evaluation (Table~\ref{tab:metrics}) and showing a wide inter-seed dispersion (Figure~\ref{fig:boxplot}).

Figure~\ref{fig:aggregate} summarizes training dynamics (moving average; mean $\pm$ std across seeds). Tabular learning reaches a stable plateau within a few thousand episodes. The \texttt{mask} representation does not improve over \texttt{simple} within our training budget; a plausible explanation is that augmenting the state with a visited-cells mask increases the effective tabular state space, which can slow down learning despite restoring Markovity in principle. Under a fixed grid layout, the lossy \texttt{simple} key may still be sufficient to learn a near-optimal policy for this task.

Beyond scalar return, domain diagnostics reveal qualitative policy differences. Q-learning yields short episodes ($\approx 4$ steps on average), suggesting it learns to collect a small, high-value set of ingredients and cook quickly (Figure~\ref{fig:steps}). The linear baseline produces significantly longer episodes ($25.02\pm13.26$ steps), higher waste ($2.18\pm1.41$), and a broader trade-off curve (Figure~\ref{fig:tradeoff}), consistent with slower/less decisive policies and suboptimal ingredient collection. Figure~\ref{fig:made} shows that the tabular policy produces a more diverse recipe mix, while the linear baseline concentrates on a smaller set of outcomes and exhibits a larger fraction of failures (no feasible recipe cooked).

Finally, Figure~\ref{fig:sample_eff} compares sample-efficiency proxies derived from training curves. Q-learning achieves higher average train return (AUC proxy) and higher late-stage performance, while the linear baseline is lower on both metrics, matching the observed evaluation gap.

\subsection{Discussion and Limitations}
These results highlight a key practical point: in small, structured discrete environments, tabular control can be both data-efficient and stable. The linear baseline here uses a simple feature vector and an off-policy TD update, which can be sensitive to feature design, exploration schedule, and reward shaping. High variance across seeds (Figure~\ref{fig:boxplot}) suggests that learning outcomes depend strongly on early trajectories and exploration decisions.

The \texttt{simple}/\texttt{mask} comparison should be interpreted carefully for function approximation. In our implementation, \texttt{state\_mode} affects only the discrete tabular key; the observation features used by the linear model are unchanged, hence \texttt{simple} and \texttt{mask} are expected to be identical for the linear baseline (as observed in Table~\ref{tab:metrics} and Figures~\ref{fig:steps}--\ref{fig:sample_eff}).

Limitations include the absence of extensive hyperparameter tuning and the simplicity of the linear approximator. Future work could (i) add the visited-cells mask (or a learned memory) to the feature representation for function approximation, (ii) evaluate on randomized grid layouts for robustness, and (iii) compare additional baselines (e.g., SARSA, eligibility traces, or more stable variants of Q-learning with function approximation \cite{TsitsiklisVanRoy1997}).

\section{Conclusion}
We presented \emph{RL Chef}, a compact Gymnasium environment designed to compare tabular control with linear value function approximation and to study the impact of state representation. Across 10 seeds, tabular Q-learning reliably achieves high return with low variability, while the linear baseline is less stable and incurs higher waste and longer episodes. The analysis illustrates how representational choices and approximation can dominate performance even in small environments, motivating careful state design and more expressive function approximation for scalable RL.


\section{Visualizations}
Figure~\ref{fig:aggregate} shows aggregated learning curves (moving average window $=200$ episodes; mean $\pm$ std across seeds). Table~\ref{tab:metrics} summarizes the aggregate evaluation metrics.

\begin{figure}[t]
\centering
\includegraphics[width=\linewidth]{rlchef/returns_aggregate.png}
\caption{Aggregated learning curves (moving average; train return) across seeds: mean $\pm$ std.}
\label{fig:aggregate}
\end{figure}

\begin{table}[t]
\centering
\caption{Aggregate evaluation metrics across seeds (mean $\pm$ std).}
\label{tab:metrics}
% Auto-generated by: python -m rlchef.analyze
\begin{tabular}{llrcccc}
\toprule
Algo & State & $n$ & Return (eval) & Steps (eval) & Waste & Incompat \\
\midrule
linear & mask & 10 & 1.652 $\pm$ 0.718 & 25.02 $\pm$ 13.26 & 2.18 $\pm$ 1.41 & 0.05 $\pm$ 0.03 \\
linear & simple & 10 & 1.652 $\pm$ 0.718 & 25.02 $\pm$ 13.26 & 2.18 $\pm$ 1.41 & 0.05 $\pm$ 0.03 \\
qlearning & mask & 10 & 3.177 $\pm$ 0.079 & 3.92 $\pm$ 0.24 & 0.68 $\pm$ 0.13 & 0.05 $\pm$ 0.02 \\
qlearning & simple & 10 & 3.307 $\pm$ 0.104 & 3.96 $\pm$ 0.25 & 0.61 $\pm$ 0.13 & 0.04 $\pm$ 0.03 \\
\bottomrule
\end{tabular}

\end{table}

\begin{figure*}[t]
\centering
\begin{minipage}[t]{0.49\textwidth}
\centering
\includegraphics[width=\linewidth]{rlchef/eval_return_boxplot.png}
\caption{Evaluation return distribution across seeds (boxplot).}
\label{fig:boxplot}
\end{minipage}\hfill
\begin{minipage}[t]{0.49\textwidth}
\centering
\includegraphics[width=\linewidth]{rlchef/steps_bars.png}
\caption{Episode length in evaluation (mean $\pm$ std across seeds).}
\label{fig:steps}
\end{minipage}
\end{figure*}

\begin{figure*}[t]
\centering
\begin{minipage}[t]{0.49\textwidth}
\centering
\includegraphics[width=\linewidth]{rlchef/waste_incompat_bars.png}
\caption{Domain diagnostics: waste and incompatibility (mean $\pm$ std across seeds).}
\label{fig:waste_incompat}
\end{minipage}\hfill
\begin{minipage}[t]{0.49\textwidth}
\centering
\includegraphics[width=\linewidth]{rlchef/tradeoff_scatter.png}
\caption{Run-level trade-off: evaluation return vs penalties (waste + incompatibility).}
\label{fig:tradeoff}
\end{minipage}
\end{figure*}

\begin{figure*}[t]
\centering
\begin{minipage}[t]{0.49\textwidth}
\centering
\includegraphics[width=\linewidth]{rlchef/made_distribution.png}
\caption{Which dishes are produced? Stacked proportions over evaluation episodes; \texttt{fail} indicates no feasible recipe cooked.}
\label{fig:made}
\end{minipage}\hfill
\begin{minipage}[t]{0.49\textwidth}
\centering
\includegraphics[width=\linewidth]{rlchef/sample_efficiency.png}
\caption{Training-curve proxies: mean train return (AUC proxy) and last-episodes mean (stability), mean $\pm$ std across seeds.}
\label{fig:sample_eff}
\end{minipage}
\end{figure*}

\FloatBarrier



% conference papers do not normally have an appendix



% use section* for acknowledgment
\ifCLASSOPTIONcompsoc
  % The Computer Society usually uses the plural form
  \section*{Disclosure Statement}
\else
  % regular IEEE prefers the singular form
  \section*{Disclosure Statement}
\fi

The author declares that this report is entirely original and does not contain any plagiarism.





% trigger a \newpage just before the given reference
% number - used to balance the columns on the last page
% adjust value as needed - may need to be readjusted if
% the document is modified later
%\IEEEtriggeratref{8}
% The "triggered" command can be changed if desired:
%\IEEEtriggercmd{\enlargethispage{-5in}}

% references section

% can use a bibliography generated by BibTeX as a .bbl file
% BibTeX documentation can be easily obtained at:
% http://mirror.ctan.org/biblio/bibtex/contrib/doc/
% The IEEEtran BibTeX style support page is at:
% http://www.michaelshell.org/tex/ieeetran/bibtex/
%\bibliographystyle{IEEEtran}
% argument is your BibTeX string definitions and bibliography database(s)
%\bibliography{IEEEabrv,../bib/paper}
%
% <OR> manually copy in the resultant .bbl file
% set second argument of \begin to the number of references
% (used to reserve space for the reference number labels box)
\begin{thebibliography}{1}

\bibitem{SuttonBarto2018}
R. S. Sutton and A. G. Barto, \emph{Reinforcement Learning: An Introduction}, 2nd ed. MIT Press, 2018. [Online]. Available: \url{http://incompleteideas.net/book/the-book-2nd.html}

\bibitem{Watkins1992}
C. J. C. H. Watkins and P. Dayan, ``Q-learning,'' \emph{Machine Learning}, vol. 8, pp. 279--292, 1992.

\bibitem{Gymnasium}
Farama Foundation, ``Gymnasium: A Standard API for Reinforcement Learning Environments,'' Accessed: Jan. 16, 2026. [Online]. Available: \url{https://gymnasium.farama.org/}

\bibitem{TsitsiklisVanRoy1997}
J. N. Tsitsiklis and B. Van Roy, ``An Analysis of Temporal-Difference Learning with Function Approximation,'' \emph{IEEE Transactions on Automatic Control}, vol. 42, no. 5, pp. 674--690, 1997.

\end{thebibliography}





% that's all folks
\end{document}


